\documentclass[a4paper,11pt,dvips]{article}

\title{Installing Player/Stage on a Ubuntu server running in VMware Player}
\author{Sten Gr\"{u}ner}
\date{03.06.2009}

%%for urls
\usepackage{hyperref} 

\begin{document}

%small command for prompt listings
\newcommand{\prompt}[1]{
	\begin{center}
	\texttt{\$#1}
	\end{center}
}

\maketitle

\section{Introduction}
In this step-by-step guide I will show you how to install player/stage on a fresh and minimal Ubuntu installation running in a virtual machine. After completing the tutorial you will be able to start working with Player/Stage and Erlang modules we have written for it. All software used here is freeware. You will need about 2.5 GB space on your hard drive.

\section{Virtual Machine}
\subsection{VMware Player}
We have tested our system using VMware Player which can be obtained here: http://www.vmware.com/products/player/. Install the VMware player.
\subsection{Ubuntu Server}
We recommend to download the pre-installed VM instead of installing Ubuntu from an .iso image. An archive with the VM can be obtained here: \\ http://chrysaor.info/?page=ubuntu. Select "Ubuntu 8.04.1 Server VMware image", unpack the archive and double click the file "Ubuntu.vmx". VMware player will start (\textbf{Important:} select "I moved it" when VMware asks) and boot Ubuntu server.

\section{Ubuntu cofiguration}
Login using user name \texttt{user} and password \texttt{user}. 
\\Obtain the configuration script made by Tom (\$ symbolizes the terminal promt): 
\prompt{wget http://kerl.svn.sourceforge.net/viewvc/kerl/trunk/install/installvm.sh}
Make the script executable:
\prompt{chmod a+x installvm.sh}
Start the script (root password is \texttt{user}):
\prompt{sudo ./installvm.sh}
The script will install xfce and update your copy of Ubuntu. Select "Continue without a valid swap space" if prompted. Do not worry if "VMware tools not found" is prompted, VMware tools are not available for VMware player.
\\
Reboot the system:
\prompt{sudo reboot}
After reboot you should see a login screen on xfce. After logging in add a task panel: right-click on the "star" you can find in the upper edge, select "Add New Item", select "Task List" and click "Add".


\section{Player/Stage}
Start a terminal window: right-click on desktop, select "Accessories" $\to$ "Terminal". Download the installation script:
\prompt{wget http://kerl.svn.sourceforge.net/viewvc/kerl/trunk/install/installps.sh}
Make the script executable:
\prompt{chmod a+x installps.sh}
Start the script:
\prompt{sudo ./installps.sh}
The script will check all packages needed for Player/Stage. Then it will install Player 2.1.2 into \texttt{/usr/share/player} and Stage 3.0.1 into \texttt{/ust/share/stage}. Source directories will stay in your home directory since they contain documentation and examples. The sctipt will also install Erlang and everything else needed to start programming.
\\
After installing, close the terminal window and open a now one: right-click on desktop, select "Accessories" $\to$ "Terminal".

\section{Test your installation}
Try to start player and stage:
\prompt{player}
\prompt{stage}
\end{document}
